\documentclass[a4paper,12pt]{scrartcl}
\usepackage[utf8]{inputenc}
\usepackage[UKenglish]{isodate}
\usepackage{csquotes}
\usepackage{graphicx}
\usepackage{wrapfig}
\usepackage{enumitem}
\usepackage{pdflscape}
\usepackage[toc,page]{appendix}
\usepackage{geometry}
\usepackage{hyperref}
\usepackage{cleveref}
\usepackage{listings}
\usepackage{csvsimple}
\usepackage{booktabs}
\usepackage{longtable}
\usepackage{caption}
\usepackage{subcaption}
\usepackage[colorinlistoftodos]{todonotes}
%\usepackage[margin=1in]{geometry}
\usepackage{listings}
\usepackage{color}
 
\definecolor{codegreen}{rgb}{0,0.6,0}
\definecolor{codegray}{rgb}{0.5,0.5,0.5}
\definecolor{codepurple}{rgb}{0.58,0,0.82}
\definecolor{backcolour}{rgb}{0.95,0.95,0.92}
 
\lstdefinestyle{mystyle}{
	language=PHP,
    backgroundcolor=\color{backcolour},   
    commentstyle=\color{codegray},
    keywordstyle=\color{magenta},
    numberstyle=\tiny\color{codegray},
    stringstyle=\color{codegreen},
    basicstyle=\footnotesize,
    breakatwhitespace=false,         
    breaklines=true,                 
    captionpos=b,                    
    keepspaces=true,                 
    numbers=left,                    
    numbersep=5pt,                  
    showspaces=false,                
    showstringspaces=false,
    showtabs=false,                  
    tabsize=3,
    morekeywords={ new, __halt_compiler, abstract, and, array, as, break, callable, case, catch, class, clone, const, continue, declare, default, die, do, echo, else, elseif, empty, enddeclare, endfor, endforeach, endif, endswitch, endwhile, eval, exit, extends, final, for, foreach, function, global, goto, if, implements, include, include_once, instanceof, insteadof, interface, isset, list, namespace, new, or, print, private, protected, public, require, require_once, return, static, switch, throw, trait, try, unset, use, var, while, xor}
}

\lstset{language=Java,
  showspaces=false,
  showtabs=false,
  breaklines=true,
  showstringspaces=false,
  breakatwhitespace=true,
  commentstyle=\color{pgreen},
  keywordstyle=\color{pblue},
  stringstyle=\color{pred},
  basicstyle=\ttfamily,
  moredelim=[il][\textcolor{pgrey}]{$$},
  moredelim=[is][\textcolor{pgrey}]{\%\%}{\%\%}
}
 
\lstset{style=mystyle}

\graphicspath{ {images/} }
\usepackage[
	backend=biber,
	style=ieee,
	]{biblatex}

\addbibresource{references.bib}

\title{Web Apps Report}
\author{Candidate No: 105936}
\date{\today}

\begin{document}
	
	\begin{titlepage}
		\maketitle
	\end{titlepage}
	
	\tableofcontents
	\newpage
	\section{3-Tier Architectural Model}
	{
		All of the Java code is in three different folders with it being split between data/entities services/logic and controllers for the views therefore it is split quite nicely. Ideally they would also be split further depending other their uses to make the code cleaner. It makes use of the Model-view-controller patter by using the xhtml pages as the views, the jsf backing beans as the controllers and the entities as the models. However I feel that the three tier could be better implemented by splitting the application across different programs. For example if we were to implement a relational database manually with a orm framework it would be a lot simpler and have performance and understanding bonuses. We could also use a different web server which focuses on using controllers to provide a restful API. and then a separate fronted which would communicate with the API. This has a number of benefits such as being able to create multiple application which communicate with the API, being able to split the code across multiple project meaning that its usages are clearer.
	}
	\section{Securing the Application}
	{
		The application has been secured using realms and then restrictions have been put on access available to the pages using the web.xml deployment descriptor. However I feel that this is an overly complicated setup and that a much more simpler solution is to manage the restrictions separately as it would be far more functional and quicker. Using cookies with a sessionId would also mean you can include other forms of authentication such as OAuth. We are also only using a self signed certificate therefore when a real user tries to access the web application the browser will throw up a warning.
	}
	\section{Extending Design to Avoid a Single Point of failure}
	{
		The main way would be to extract the database to a different server then ideally dockerise the application and then use kubernetes/docker swarm to deploy the application on multiple servers or deploy multiple containers of the application on one server. To do this you need to make sure that none of the requests are session based as different servers can pick up different requests. This is where a restful API also has benefits as it shouldn't matter which server receives a request as they will all handle it the same. 
	}
	\section{Concurrency}
	{
		currently the server shouldn't have any problems with concurrency as the entity manager handles communication with the database and transactions shouldn't overlap. However it is also possible to put locks on objects to make sure that you are the only person editing it. 
	}
	
	\newpage
	
	\printbibliography[heading=bibintoc,title=References]
\end{document}
